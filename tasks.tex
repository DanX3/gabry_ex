\documentclass{article}
\usepackage{mathtools}
\usepackage{verbatim}
\begin{document}
\title{Esercizi Programmazione}
\maketitle

\section*{Impostazione}
Il programma e' strutturato in modo da lanciare automaticamente i programmi gia'
implementati e testarli con qualche input che ho scelto. \\
Usando il comando \textit{make} potrai lanciare il programma che vuoi testare.
Ad esempio
\begin{verbatim}
    make 3
\end{verbatim}
compilera' il file \textit{ex03.java} e cerchera' di eseguirlo con vari test da
me scelti. Considerero' un programma completo se ha passato tutti i test \\

\section*{Descrizione del problema}
Troverai tutti i file nella directory \textit{tasks} con gia' la classe
principale implementata e la descrizione del problema. \\
Questa avra' un titolo e la consegna. l'ultima riga rappresentera' l'input che
dovresti aspettarti con le seguenti convenzioni
\begin{itemize}
    \item \textbf{int} per input intero
    \item \textbf{long} per input intero in doppia lunghezza
    \item \textbf{double} per input reale
    \item \textbf{string} per input simbolico
\end{itemize}
Se non e' presente nessun parametro, il programma non deve leggere alcun parametro

\section*{Soluzioni}
Nel caso non riuscissi a risolvere il problema, ho fornito una mia soluzione
nella directory \textit{solutions}. Usala come ultima risorsa. Usa allo stesso
modo il makefile e osserva le soluzioni da me fornite. Anche se il programma
termina con successo, alcuni risultati potrebbero differire dai tuoi.
Fai attenzione e in bocca al lupo!

\end{document}
