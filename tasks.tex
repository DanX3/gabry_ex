\documentclass{article}
\usepackage{mathtools}
\usepackage{verbatim}
\begin{document}
\title{Esercizi Programmazione}
\maketitle

\section*{Impostazione}
Il programma e' strutturato in modo da lanciare automaticamente i programmi gia'
implementati e testarli con qualche input che ho scelto. \\
Usando il comando \textit{make} potrai lanciare il programma che vuoi testare.
Ad esempio
\begin{verbatim}
    make 3
\end{verbatim}
compilera' il file \textit{ex03.java} e cerchera' di eseguirlo con vari test da
me scelti. Considerero' un programma completo se ha passato tutti i test \\

\section*{Descrizione del problema}
Troverai tutti i file nella directory \textit{tasks} con gia' la classe
principale implementata e la descrizione del problema. \\
Questa avra' un titolo e la consegna. l'ultima riga rappresentera' l'input che
dovresti aspettarti con le seguenti convenzioni
\begin{itemize}
    \item \textbf{int} per input intero
    \item \textbf{long} per input intero in doppia lunghezza
    \item \textbf{double} per input reale
    \item \textbf{string} per input simbolico
\end{itemize}
Se non e' presente nessun parametro, il programma non deve leggere alcun parametro

\section*{Soluzioni}
Nel caso non riuscissi a risolvere il problema, ho fornito una mia soluzione
nella directory \textit{solutions}. Usala come ultima risorsa. Usa allo stesso
modo il makefile e osserva le soluzioni da me fornite. Anche se il programma
termina con successo, alcuni risultati potrebbero differire dai tuoi.
Fai attenzione e in bocca al lupo!


\section*{Esercizi Java}
Di seguito elechero' una serie di esercizi riguardante la Programmazione a
oggetti (OOP). Questi richiedono una maggiore conoscenza di Java e
possono essere in realta' essere risolti in modi diversi, non usando comunque la
OOP, percio' cerca di usare il paradigma ad oggetti ove possibile

\subsection*{Ereditarieta'}
\begin{verbatim}
Object
    AbstractCollection
        AbstractList
            ArrayList
\end{verbatim}
Un elemento chiave della OOP e' la capacita' di creare superclassi estese di
volta in volta in base al problema da risolvere. Cio' porta all'efficiente
propagazione di una modifica verso tutte le sottoclassi: se in una nuova
versione di Java si decide che un oggetto debba avere un parametro aggiuntivo,
modificando la classe \textit{Object} si propaga la modifica a tutto cio' che e'
un oggetto (in questo caso veramente qualsiasi cosa). Cosi' come i miglioramenti
si propagano anche gli errori, quindi parti disegnando bene il tuo sistema. \\
Cerchiamo di esplorare l'efficienza dell'ereditarieta'.

\subsection*{Polimorfismo}
Attraverso le interfacce, e' possibile scrivere una serie di metodi che una
classe deve avere. Una classe puo' implementare piu' interfacce ed e' quindi
meno vincolante dell'ereditarieta' che la limita a una. Vedila piu'
semplicemente come un pezzo di un puzzle che posso attaccare alla classe
\textit{``Se mi implementi questi metodi, ti posso considerare come
quell'interfaccia''} \\
Inoltre una stessa interfaccia puo' essere implementata da piu' classi. Tutte
queste classi percio' possono essere considerate come quella stessa interfaccia,
ma sono effettivamente diverse. Un esempio sara' piu' chiaro:
\begin{verbatim}
Shape
 -> Circle
 -> Square
 -> Triangle
\end{verbatim}
Tutte le forme, Circle, Square e Triangle sono classi diverse che implementano
la stessa interfaccia \textit{Shape}. E' possibile comunque avere un array di
Shape e popolarlo con tutte e 3 le classi.
\begin{verbatim}
Shape s = new Triangle();
\end{verbatim}
e' una sintassi perfettamente valida, poiche' Triangle puo' essere
effettivamente visto come Shape.

\newpage

\subsection*{Esercizio 1}
Implementare una classe \textit{Person} strutturata come segue
\begin{verbatim}
class Person
    Person(String name, String surname)
    String getName()
    String getSurname()
    void sayHi()
\end{verbatim}
dove sayHi() stampi a schermo il nome e il cognome della persona.
 
In seguito scrivi una classe \textit{Student} che includa anche un campo intero
\textit{badge\_number} con relativo getter (il metodo che restituisce il
campo).\\
Sempre ereditando da \textit{Person} crea anche una classe \textit{Teacher} con
un array di stringhe dei corsi di cui di occupa. \\
Nel main crea un array di \textit{Person} e popolalo con 2 persone, 3 studenti e
un insegnante. Tutti dovranno dire il proprio nome. \\
Ipotizza che passi del tempo e l'idea di persona non ti soddisfa piu': ti serve
anche un campo che ne registri il codice fiscale. Effettua le dovute modifiche
in modo che anche le sottoclassi siano preparate a ricevere un codice fiscale in
input (crea altri costruttori o modifica l'originale). \\ \\
\begin{itemize}
    \item Nota che in tutte le tue istanze e' presente il metodo \textit{sayHi()} grazie
all'ereditarieta'. Nota anche che dopo aver fatto una modifica a un costruttore,
sei costretto a rimaneggiare tutte le successive chiamate nella tua code base.
Per piccoli progetti non e' un problema, ma quando lavori con decine di migliaia
di righe di codice, preferiresti spendere il pomeriggio a lavorare a una nuova
feature piuttosto che ripassare tutte le classi del progetto in cerca delle
chiamate ai costruttori ;)
\end{itemize}


\end{document}
